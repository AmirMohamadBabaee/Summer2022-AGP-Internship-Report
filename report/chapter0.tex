\chapter{مقدمه}
صوت یکی از مهم‌ترین حالات انرژی در جهان ما می‌باشد و راه ارتباطی اصلی بسیاری از انسان‌ها و دیگر موجودات،‌ از طریق سیگنال های صوتی می‌باشد. به همین دلیل، درک و پردازش این نوع از داده‌ها، اهمیت بسیاری در عصر حاضر برای ما دارا می‌باشد. 

علاوه بر این، یکی از بهترین حالات تعامل انسان با رایانه، استفاده از صوت و دستورات گفتاری است. بنابراین، برای دستیابی به چنین قابلیتی، نیاز است که گفتار برای رایانه‌ها قابلیت پردازش و درک پیدا کرده و سپس از آن برای برقراری ارتباط راحت‌تر میان انسان و رایانه استفاده کرد.

برای این کار،‌امروزه سامانه‌ها و مدل‌هایی وجود دارند که صرفا بر روی قسمت صوتی گفتار متمرکز می‌باشند. این مدل‌ها با اینکه در موقعیت‌های عادی و بدون نویز، به دقت و عملکرد مناسبی دست پیدا کرده‌اند، اما در موقعیت‌های نویزی و با کیفیت پایین، عملکرد نسبتا ضعیفی از خود نشان می‌دهند و به همین دلیل مدل‌های قابل اتکایی نمی‌باشند.

یکی از راهکار‌ها برای قابل‌اتکا کردن این نوع از مدل‌ها، استفاده از داده‌های کمکی می‌باشد. یک نمونه از این داده‌های کمکی، داده های تصویری حرکت لب‌های فرد گوینده می‌باشد. این داده‌ها قابلیت جبران نقص اطلاعات سیگنال‌های صوتی را دارا می‌باشند. 

این نوع از عملکرد، معادل عملکرد سیستم شنیداری انسان نیز می‌باشد. در انسان نیز، با اینکه گوش، مهم‌ترین نقش را ایفا می‌کند، اما تنها مولفه نمی‌باشد. این موضوع زمانی واضح‌تر می‌شود که در یک محیط شلوغ، به دنبال درک جملات بیان شده توسط یک گوینده هستیم. در این حالت، حرکت لب‌های فرد در کنار گفتار ضعیفی که از فرد به ما می‌رسد، در کنار هم منجر به درک درست گفتار بیان شده فرد گوینده از سمت ما می‌شود.

علاوه بر این موضوع، برای ساخت و پیاده‌سازی چنین سامانه‌هایی، یکی از مهم‌ترین ارکان، وجود داده‌های آموزشی می‌باشد. این نوع از داده‌ها، در زبان‌هایی نظیر زبان انگلیسی به نسبت، به مقدار بیشتری وجود دارند این در حالی است که در زبان فارسی حجم دادگان‌های موجود به نسبت، کم‌تر می‌باشد. یکی از مواردی که در این گزارش در رابطه با آن صحبت خواهد شد، روش جمع‌آوری و گردآوری یک دادگان صوتی-تصویری برای ارائه و استفاده در حل مساله بازشناسی گفتار به واسطه صوت و تصویر می‌باشد.
\\

در ادامه، در فصل دوم به معرفی شرکت عصرگویش‌پرداز پرداخته و بخشی از مهم‌ترین محصولات و زمینه‌های فعالیت این شرکت بررسی خواهند شد. در فصل سوم، تجربیات کسب شده در این دوره کارآموزی سه‌ماهه، بیان خواهد شد و برخی از چالش‌ها و راه‌حل‌هایی که در این دوره ارائه شدند، بررسی خواهند شد. در نهایت در فصل چهارم، نتیجه‌گیری مربوط به این دوره کارآموزی بیان خواهد شد و پیشنهاد‌هایی در جهت بهبود مدل و دادگان ارائه شده، ذکر خواهد شد.
