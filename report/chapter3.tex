\chapter{نتیجه‌گیری و پیشنهاد‌ها}

در این فصل، در ابتدا به مرور نکات ذکر شده و جمع‌بندی آنها پرداخته و سپس پیشنهاد‌هایی در جهت بهبود و ارتقا سامانه و دادگان جمع‌آوری شده ارائه می‌شود.

\section{نتیجه‌گیری و جمع‌بندی}

همانطور که در فصل‌های قبل بررسی شد، استفاده از داده‌های تصویری حرکت لب‌های فرد گوینده، داده مناسبی برای جبران نقص در سیگنال‌های صوتی مربوط به گفتار می‌باشد. علاوه بر این، این مکانیزم در سیستم شنیداری انسان نیز وجود دارد و علاوه بر گوش‌ها، دیدن حرکت لب‌های گوینده نیز تاثیر به سزایی در فهم متن بیان شده توسط گوینده دارد.

همچنین، به دلیل محدود بودن داده‌های برچسب‌گذاری شده ویدیویی، استفاده از رویکرد‌های خود-نظارتی و نیمه-نظارت‌شده نسبت به رویکرد نظارت‌شده، عملکرد بهتری از خود نشان داده و از پایداری بهتری برخوردار خواهد بود. در این مدل‌ها، تلاش می‌شود که دانش کلی نسبت به ماهیت و ارتباط داده‌ها به دست آمده (در فرایند پیش‌آموزش بر روی داده‌های بدون برچسب) و سپس این دانش به طور خاص بر روی حل مساله مورد نظر کوک شود.

این رویکرد، رویکرد مناسبی برای استفاده در زبان‌هایی است که داده ویدیو کافی نداشته باشند؛ چراکه با وجود داده برچسب‌گذاری شده کم نیز، توانایی رسیدن به عملکرد و دقت مناسب را دارا می‌باشند. با این حال، در زبان فارسی داده ویدیویی مناسب برای این مساله موجود نمی‌باشد. به همین دلیل در این مقاله در پی این بر آمدیم تا دادگان ویدیویی فارسی با استفاده از ویدیو‌های آرشیو شبکه خبر صدا و سیما جمهوری اسلامی ایران جمع‌آوری کنیم.


\section{پیشنهاد‌ها}

از جمله پیشنهاد‌هایی که می‌توان در جهت بهبود دادگان فعلی داد، استفاده از ویدیو برنامه‌های گفتگومحور دیگر شبکه‌های صدا و سیما جمهوری اسلامی ایران می‌باشد. علاوه بر این در صورت توسعه یک خط لوله پردازشی دقیق‌تر برای حذف خروجی‌های اشتباه مدل سینک‌نت، می‌توان به دادگان با کیفیت بالاتری دست یافت. این دادگان به دلیل حجم تخمینی کمی که دارد، برای فرایند کوک کردن مدل‌های بازشناسی گفتار به کمک صوت و تصویر، مناسب می‌باشد اما برای اجرای فرایند پیش‌آموزش مناسب نمی‌باشد. یکی از دیگر پیشنهاد‌ها در جهت بهبود این دادگان، استفاده از دیگر سایت‌های اشتراک‌گذاری ویدیو آنلاین نظیر آپارات، فرادرس و مکتب‌خونه می‌باشد.

