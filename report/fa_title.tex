%% -!TEX root = AUTthesis.tex
% در این فایل، عنوان پایان‌نامه، مشخصات خود، متن تقدیمی‌، ستایش، سپاس‌گزاری و چکیده پایان‌نامه را به فارسی، وارد کنید.
% توجه داشته باشید که جدول حاوی مشخصات پروژه/پایان‌نامه/رساله و همچنین، مشخصات داخل آن، به طور خودکار، درج می‌شود.
%%%%%%%%%%%%%%%%%%%%%%%%%%%%%%%%%%%%
% دانشکده، آموزشکده و یا پژوهشکده  خود را وارد کنید
\faculty{دانشکده مهندسی کامپیوتر}
% گرایش و گروه آموزشی خود را وارد کنید
\department{محل کارآموزی: شرکت عصر گویش پرداز}
% عنوان پایان‌نامه را وارد کنید
%\fatitle{مقدمه ای بر تشخیص خشونت در نظارت ویدیویی به کمک یادگیری عمیق}
% نام استاد(ان) راهنما را وارد کنید
\firstsupervisor{دکتر احمد نیک آبادی}
\secondsupervisor{}
% نام استاد(دان) مشاور را وارد کنید. چنانچه استاد مشاور ندارید، دستور پایین را غیرفعال کنید.
%\firstadvisor{نام کامل استاد مشاور}
%\secondadvisor{استاد مشاور دوم}
% نام نویسنده را وارد کنید
\name{امیرمحمد}
% نام خانوادگی نویسنده را وارد کنید
\surname{بابائی}
%%%%%%%%%%%%%%%%%%%%%%%%%%%%%%%%%%
\thesisdate{تابستان ۱۴۰۱}

% چکیده پایان‌نامه را وارد کنید
\fa-abstract{
	امروزه، مدل‌های تبدیل گفتار به متن، کاربرد بسیاری در چت‌بات‌ها، تعامل ساده‌تر انسان با رایانه و شبکه‌های پخش آنلاین ویدیو پیدا کرده‌اند. با این حال، مدل‌های فعلی در موقعیت‌های نویزی و دارای کیفیت پایین، عملکرد ضعیفی از خود نشان می‌دهند. یکی از رویکرد‌های اصلی برای بهبود عملکرد این سیستم‌ها، استفاده از داده‌های کمکی غیر از سیگنال‌های صوتی می‌باشد. از نمونه این نوع داده‌های کمکی، داده‌های تصویری حرکت لب‌های گوینده می‌باشد که توانایی جبران نقص اطلاعات داده‌های صوتی مربوط به گفتار را دارا می‌باشد. رویکرد غالب در این روش، استفاده از روش‌های نظارت‌شده
	\LTRfootnote{Supervised}
	می باشد اما با این حال، به دلیل محدود بودن داده‌های برچسب گذاری شده صوتی-تصویری، روش مناسبی در حال حاضر نمی‌باشند. رویکرد جدید برای دستیابی به نتایج بهتر، استفاده از مدل‌های خود-نظارتی
	\LTRfootnote{Self-Supervised}
	می باشد که توانایی رسیدن به عملکرد مناسب با استفاده از حجم داده برچسب‌گذاری شده کمتر را دارا هستند. مدل ای-وی هیوبرت
	\LTRfootnote{AV-HuBERT}
	نمونه ای از مدل‌های مبتنی بر این رویکرد می‌باشد. علاوه بر این، به دلیل نبود دادگان مناسب برای حل این مساله در فارسی، مراحل جمع‌آوری یک دادگان صوتی-تصویری فارسی با استفاده از ویدیو‌های آرشیو سایت تلوبیون نیز در این گزارش، ذکر شده است.
}


% کلمات کلیدی پایان‌نامه را وارد کنید
\keywords{
	بازشناسی گفتار، بازشناسی گفتار به واسطه صوت و تصویر، دادگان صوتی-تصویری
}



\AUTtitle
%%%%%%%%%%%%%%%%%%%%%%%%%%%%%%%%%%
\vspace*{7cm}
\thispagestyle{empty}
\begin{center}
\includegraphics[height=5cm,width=12cm]{besm}
\end{center}
\AUTtitle